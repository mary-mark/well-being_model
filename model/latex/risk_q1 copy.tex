\documentclass[10pt]{article}
\usepackage[usenames]{color} %used for font color
\usepackage{amssymb} %maths
\usepackage{amsmath} %maths
\usepackage[utf8]{inputenc} %useful to type directly diacritic characters
\usepackage{booktabs}
\begin{document}
\[\begin{tabular}{@{}lrrr@{}}
\toprule
& & \multicolumn{2}{c}{\bf Per capita risk}\\
{} &  \bf Q1 population &  Asset &  Well-being \\
\bf Region  & \footnotesize [thousands]  &  \multicolumn{2}{@{}c@{}}{\footnotesize [PhP per cap., per year]}\\
\midrule
V - Bicol                &  1,206 &         544 &      11,874 \\
VIII - Eastern Visayas   &    906 &         312 &       6,456 \\
II - Cagayan Valley      &    698 &         755 &       6,218 \\
I - Ilocos               &  1,024 &         509 &       5,705 \\
IVA - CALABARZON         &  2,825 &         464 &       5,395 \\
III - Central Luzon      &  2,219 &         431 &       5,354 \\
VII - Central Visayas    &  1,487 &         161 &       4,001 \\
CAR                      &    356 &         157 &       2,888 \\
XIII - Caraga            &    542 &         115 &       2,501 \\
VI - Western Visayas     &  1,539 &         137 &       2,417 \\
NCR                      &  2,528 &         471 &       2,388 \\
IVB - MIMAROPA           &    615 &         135 &       2,224 \\
XII - SOCCSKSARGEN       &    913 &          31 &       1,043 \\
XI - Davao               &    991 &          64 &         934 \\
ARMM                     &    739 &          32 &         599 \\
X - Northern Mindanao    &    940 &          27 &         535 \\
IX - Zamboanga Peninsula &    751 &          22 &         281 \\
\midrule
All regions            &  20,278 &        308 &       3,943 \\
\bottomrule
\end{tabular}\]
\end{document}